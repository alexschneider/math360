%        File: homework1.tex
%     Created: Tue Jan 20 02:00 PM 2015 P
% Last Change: Tue Jan 20 02:00 PM 2015 P
%
\documentclass[11pt]{article}
\usepackage{geometry}
\geometry{letterpaper}
\usepackage[parfill]{parskip}
\usepackage{booktabs}
\title{MATH 360 Homework 1}
\author{Alex Schneider}
\begin{document}
\maketitle
\section*{Section 1.1}
\subsection*{Problem 2}
For estimating the mean height of all students in a university, the best
sampling strategy would be option \textit{iii}, measuring the height of the first
student on each page of the phone book. This gives a more pseudorandom sample
than the other two options, because the population is likely to be more diverse
than all the basketball players or all the engineering majors, and less
susceptible to sample bias. 

\subsection*{Problem 3}
\subsubsection*{a.}
False. A simple random sampling is extremely likely, given a large enough sample
size, to provide a representative sample, but there is still a chance that it
may be biased one way or another, due to the nature of randomness. 

\subsubsection*{b.}
True. A simple random sample is free from any sort of tendency towards any bias,
every member of a population is equally likely to be picked.

\subsection*{Problem 4}
\subsubsection*{a.}
False. Random sampling isn't perfect.

\subsubsection*{b.}
True.

\section*{Section 1.2}
\subsection*{Problem 1}
False. The mean is the sum of all numbers divided by the number of numbers is
the mean. If the numbers are evenly distributed, the average might be the exact
middle, but if there are any outliers, it is unlikely that this will be the
case.

\subsection*{Problem 2}
The sample mean is not always the most frequently occurring value. For example,
consider the data in Figure~\ref{fig:example1}.

\begin{figure}[h]
    \centering
    \begin{tabular}{|l|c|r|} % chktex 44
        \hline % chktex 44
        1 & 1 & 1 \\
        \hline % chktex 44
        4 & 5 & 7 \\
        \hline % chktex 44
        8 & 9 & 9 \\
        \hline % chktex 44
    \end{tabular}
    \caption{Example data for Problem 2}
\label{fig:example1}
\end{figure}


The mean is 5, but the most frequently occurring value is 1.

\subsection*{Problem 3}
The sample mean is not always equal to one of the values in the sample. For
example, consider the data in Figure~\ref{fig:example2}. 

\begin{figure}[h]
    \centering
    \begin{tabular}{|l|c|r|} % chktex 44
        \hline % chktex 44
        1 & 1 & 1 \\
        \hline % chktex 44
        4 & 5 & 6 \\
        \hline % chktex 44
        7 & 9 & 9 \\
        \hline % chktex 44
    \end{tabular}
    \caption{Example data for Problem 3}
\label{fig:example2}
\end{figure}

The mean is $4.\overline{8}$, which is not in the set. 

\subsection*{Problem 4}
The sample median is not always equal to one of the values in the sample. If
there are an even number of values in the sample, the sample median is equal to
the value at the $\frac{n}{2}$ and $\frac{n}{2} + 1$ positions averaged. For
example, consider the data in Figure~\ref{fig:example3} 

\begin{figure}[h]
    \centering
    \begin{tabular}{|l|c|r|} % chktex 44
        \hline % chktex 44
        1 & 1 & 1 \\
        \hline % chktex 44
        4 & 5 & 6 \\
        \hline % chktex 44
    \end{tabular}
    \caption{Example data for Problem 4}
\label{fig:example3}
\end{figure}

The median is $\frac{1+4}{2}$ or $2.5$, which is clearly not in the sample. 

\subsection*{Problem 5}
The sample size for which the median will always equal one of the values in the
sample is any odd sample size. For example, samples with sizes of $1, 3, 5, \dots$
all have medians where the value is inside the sample itself

\subsection*{Problem 6}
The sample standard deviation may be greater than the mean if the data is all
very close together. For example, consider the data in
Figure~\ref{fig:example4}.

\begin{figure}[h]
    \centering
    \begin{tabular}{|l|c|r|} % chktex 44
        \hline % chktex 44
        1 & 1 & 1 \\
        \hline % chktex 44
        1 & 1 & 1 \\
        \hline % chktex 44
    \end{tabular}
    \caption{Example data for Problem 6}
\label{fig:example4}
\end{figure}

The mean is $1$, while the standard deviation is $0$. 

\subsection*{Problem 9}
Previously, the mean was calculated by using the formula: 
$\overline{x}=\frac{X_{0}+X_{1}+X_{2}+\cdots+X_{n}}{n}$, where $X_{i}$ was 
each employee's individual salary, and $n$ was the number of employees
working at the company. To compute the new salary, we can use  
$\overline{x}=\frac{1.05 \times X_{0}+1.05 \times X_{1}+1.05 \times X_{2}+\cdots+1.05 \times X_{n}}{n}$
to calculate the new mean. We can then factor out the $1.05$ to get 
$\overline{x}=1.05 \times \frac{X_{0}+X_{1}+X_{2}+\cdots+X_{n}}{n}$, which means that
the new mean is exactly $1.05$ times the pre-raise mean. 

For the standard deviation, we have the equation
$\sqrt{\frac{1}{n-1}\sum\limits_{i=1}^{n}{(X_i-\overline X)}^2}$ which gives us
the standard deviation prior to the raise, and after the raise, we can use the
equation $\sqrt{\frac{1}{n-1}\sum\limits_{i=1}^n{(1.05 \times X_i-1.05 \times \overline X)}^2}$ 
which gives us the new standard deviation. Since we can factor out a 1.05, we
are left with $\sqrt{\frac{1}{n-1} \times 1.05^2 \times \sum\limits_{i=1}^n{(X_i-\overline X)}^2}$ 
which, when we take the $1.05$ outside the square root, we get
$1.05 \times \sqrt{\frac{1}{n-1}\sum\limits_{i=1}^n{(X_i-\overline X)}^2}$, or exactly
$1.05$ times the standard deviation of the pre-raise standard deviation. 

\subsection*{Problem 10}
\subsubsection*{a.}
We can find the mean by adding all the results together and dividing by the
number of responses like so:

\[\overline X = \frac{0 \times 27+1 \times 22+2 \times 30+3 \times 12+4 \times 7+5 \times 2}{100}\]
\[\overline X = \frac{0+22+60+36+28+10}{100}\]
\[\overline X = \frac{156}{100}\]
\[\overline X = 1.56\]

The sample mean of all the children is $1.56$ children per woman.

\subsubsection*{b.}
We can find the standard deviation of the number of children like so:

\[s = \sqrt{\frac{1}{100-1}\sum\limits_{i=1}^{n}{(X_i-\overline X)}^2}\]
\[s = \sqrt{
    \frac{1}{99}
    \left(\begin{array}{c}\left( 27 \times {(0-1.56)}^2 \right) + 
        \left(22 \times {(1-1.56)}^2 \right) + \left( 30 \times {(2-1.56)}^2 \right) +\\
        \left( 12 \times {(3-1.56)}^2 \right) + \left( 7 \times {(4-1.56)}^2 \right) + 
        \left( 2 \times {(5-1.56)}^2 \right)
    \end{array}\right)
    }
\]
\[s = \sqrt{\frac{1}{99}\left(\left(22 \times
    {(-0.56)}^2\right) + \left(30 \times 1.44^2\right) + \left(12 \times 2.44^2\right)
    + \left( 7 \times 3.44^2\right) + \left(2 \times 4.44^2\right)\right) }\]
\[s = \sqrt{\frac{1}{99}\left(262.8128\right)}\]
% TODO: show more steps if there's time
\[s = \sqrt{2.6546\overline{74}}\]
\[s \approx 1.629317\] 

The standard deviation of all the children is approximately $1.629317$. 

\subsubsection*{c.}
The sample median of the number of children is the average of the two numbers at
the $\frac{100}{2}$ and $\frac{100}{2} + 1$ position, or the $50$th and $51$st
position. We know that the $28$th position is the first 1 child mother, the $50$th
position is the first two child mother, and the $80$th position position is the
first three child mother. Therefore, we know that the $50$th and $51$st position
mothers both have two children. The average of $2$ and $2$ is $\frac{2+2}{2}$ or
$2$. Therefore, the sample median number of children is $2$.

\subsubsection*{d.}a
The first quartile is found by averaging the floor and ceiling of the number
given by $0.25 \times (n + 1)$, or $0.25 \times (101)$ which gives us $25.25$.
Thus, the first quartile is the average of the $25$th and $26$th mother, which
have $0$ and $0$ children respectively. Averaging the two gives us
$\frac{0+0}{2}$ or $0$. 

\subsubsection*{e.}
The mean number of children is $1.56$, so the number of women that had more
than the mean number of children is the number of women who had more than $1.56$
children. The ratio of them is $\frac{30+12+7+2}{100}$, or
$\frac{51}{100}$, which evaluates to $0.51$, giving us a percentage of $51\%$.

\subsubsection*{f.}
The mean number of children is $1.56$, and the standard deviation is
approximately $1.629317$. Adding the two numbers together gives us approximately
$3.189317$. Thus, the number of women who had more children than one
standard deviation above the mean is those who had at least 4 children. The
ratio of them is $\frac{7+2}{100}$ or $\frac{9}{100}$, which evaluates to
$0.09$, giving us a percentage of $9\%$.

\subsubsection*{g.}
The mean number of children is $1.56$, and the standard deviation is
approximately $1.629317$. Adding the two numbers gives us approximately
$3.189317$. Thus, the number of women who had children within the standard
deviation is those who had $2$ or $3$ children. The ratio of them is
$\frac{30+12}{100}$, or $\frac{42}{100}$, which evaluates to $0.42$, giving us a
percentage of $42\%$. 

\subsection*{Problem 12 (Method C Only)} % (a-e for method c only; f; g)
\subsubsection*{Data Set}
\[
    20.2, 20.5, 20.5, 20.7, 20.8, 20.9, 21.0,21.0, 21.0, 21.0, 21.0, 21.5, 21.5,
    21.5, 21.5, 21.6
\]
\subsubsection*{a.}
The mean measurement is calculated as follows:

\[\overline{X} = \frac{\left(
    \begin{array}{c}
        20.2+20.5+20.5+20.7+20.8+20.9+21.0+21.0+\\
        21.0+21.0+21.0+21.5+21.5+21.5+21.5+21.6
    \end{array}
\right)}{16}\]

\[\overline{X} = \frac{336.2}{16}\]
\[\overline{X} = 21.0125\]

\subsubsection*{b.}
The median measurement is the average of the measurements at the $\frac{16}{2}$
and $\frac{16}{2}+1$ positions, which is the $8$th and $9$th position
respectively. The numbers at the $8$th and $9$th position are both $21.0$, which
means that the average is $21.0$, which means that the median is $21.0$. 

\subsubsection*{c.}
The $20\%$ trimmed measurement can be found by taking $20\%$ of $16$, which is
$0.2 \times 16$ which simplifies to $3.2$. We then round it up to $4$, and drop
4 measurements off of each side, and recalculate the mean. We get:

\[\overline{X} = \frac{\left(
        20.8+20.9+21.0+21.0+21.0+21.0+21.0+21.5
    \right)}{8}\]
\[\overline{X} = \frac{168.2}{8}\]
\[\overline{X} = 21.025\]

\subsubsection*{d.}
The first and third quartile is found by averaging the floor and ceiling of the
number given by $0.25 \times (16 + 1)$ and $0.75 \times (16 + 1)$ respectively. 
These simplify to $4.25$ and $12.75$ respectively, which gives us the average of
the number at the $4$th and $5$th position for the first quartile, and the
average of the number at the $12$th and $13$th position for the third quartile. 

For the first quartile, we have the average of $20.7$ and $20.8$, which is
$20.75$. For the third quartile, we have the average of $21.5$ and $21.5$, which
is $21.5$. 

\subsubsection*{e.}
We can find the standard deviation like so:

\[s = \sqrt{\frac{1}{16-1}\sum\limits_{i=1}^{n}{(X_i-\overline X)}^2}\]
\[s = \sqrt{
    \frac{1}{15}
    \left(\begin{array}{c} 
        {\left(20.2-21.0125\right)}^2+
        {\left(20.5-21.0125\right)}^2+
        {\left(20.5-21.0125\right)}^2+
        {\left(20.5-21.0125\right)}^2+\\
        {\left(20.7-21.0125\right)}^2+
        {\left(20.8-21.0125\right)}^2+
        {\left(20.9-21.0125\right)}^2+
        {\left(21.0-21.0125\right)}^2+\\
        {\left(21.0-21.0125\right)}^2+
        {\left(21.0-21.0125\right)}^2+
        {\left(21.0-21.0125\right)}^2+
        {\left(21.0-21.0125\right)}^2+\\
        {\left(21.5-21.0125\right)}^2+
        {\left(21.5-21.0125\right)}^2+
        {\left(21.5-21.0125\right)}^2+
        {\left(21.5-21.0125\right)}^2+\\
        {\left(21.6-21.0125\right)}^2
    \end{array}\right)
    }
\]
\[s \approx \sqrt{\frac{1}{15} \times 2.9}\]
\[s \approx 0.11 \]

\subsubsection*{f.}
The largest standard deviation might be \textit{Method A}, because there are no
hard and fast measurements for it. It was basically down to an educated guess,
and humans are not very good at guessing. 

\subsubsection*{g.}
It is better for a measurement to have a smaller standard deviation than a
larger deviation. It gives authenticity to the accuracy of measuring, and people
who use it are likely to get a closer measurement to the actual number they are
measuring for.

\subsection*{Problem 15}
\subsubsection*{Data Set}
\[2,18,23,41,44,46,49,61,62,74,76,79,82,89,92,95\]

\subsubsection*{a.}
The tertiles can be found by using the cut points $(\frac{1}{3})(n+1)$ and 
$(\frac{2}{3})(n+1)$. For the given data set, we have cut points $(\frac{1}{3})(17)$, which
simplifies to $5.\overline{6}$, and $(\frac{2}{3})(17)$, which simplifies to
$11.\overline{3}$. 

The data set that corresponds to the first tertile is:

\[2,18,23,41,44\]

The second tertile:

\[46,49,61,62,74,76\]

The third tertile:

\[79,82,89,92,95\]

\subsubsection*{b.}
The quintiles can be found by using the cut points $(\frac{1}{5})(n+1)$,
$(\frac{2}{5})(n+1)$,
$(\frac{3}{5})(n+1)$, $(\frac{4}{5})(n+1)$. These are  $(\frac{1}{5})(17)$,
$(\frac{2}{5})(17)$, $(\frac{3}{5})(17)$, and
$(\frac{4}{5})(17)$, respectively, which simplify down to $3.4$, $6.8$, $10.2$, and
$13.6$ respectively. 

The data set that corresponds to the first quintile is:

\[2,18,23\]

The second quintile:

\[41,44,46\]

The third quintile:

\[49,61,62,74\]

The fourth quintile:

\[76,79,82\]

The fifth quintile:

\[89, 92, 95\]

\subsection*{Problem 16}
\subsubsection*{a.}
It is unlikely that the outlier is correct, as the same rod was measured all
five times. 

\subsubsection*{b.}
The outlier could be correct, as it's not the same car being priced. One of the
cars could be a luxury or sports car. 

\end{document}

