%        File: homework3.tex
%     Created: Fri Jan 30 04:00 PM 2015 P
% Last Change: Fri Jan 30 04:00 PM 2015 P
%
\documentclass[11pt]{article}
\usepackage{geometry}
\geometry{letterpaper}
\usepackage[parfill]{parskip}
\title{Math 360 Homework 3}
\author{Alex Schneider}
\begin{document}
\maketitle
\section*{Section 2.2}
\subsection*{Problem 1}
\subsubsection*{a.}
Because this is ordered, with replacement, there are $4^3$ or $64$ different codons. 

\subsubsection*{b.}
We need to find how many codons there are who's first and third base is either
\textit{A} or \textit{G} which can be expressed as $2^2$ or $4$, and the second base is 
\textit{C} or \textit{T}, which can be expressed as $2^1$ or $2$. If we multiply
the two probabilities together, we get a total of $8$ different codons that meet
the requirements.

\subsubsection*{c.}
Because this is ordered, without replacement, there are $\frac{4!}{(4-3)!}$
different codons, which simplifies to $24$ different codons. 

\subsection*{Problem 4}
The way that we can solve this problem is by breaking it up into two parts.
First, the number of ways to divide 10 players into two teams. This is
unordered, without replacement, or $10 \choose 5$, which reduces to $252$
combinations. Then, each combination has two options, a red uniform or a blue
uniform. We multiply $252$ by $2$ to get $454$ different ways of dividing 10
basketball players in two teams. 

\subsection*{Problem 6}
In this circumstance, we have ordered, without replacement because the
president, vice president, and secretary cannot be the same person. We 
can express this by the formula $\frac{8!}{(8-3)!}$, which simplifies to $336$
different ways to chose a president, vice president, and secretary from an 8
person committee. 

\subsection*{Problem 9}
\subsubsection*{a.}
There are $26$ letters and $10$ digits, for a total of $36$ possible
options for each character. Since characters can be chosen more than once,
this is ordered, with replacement. Therefore, there are $36^8$, or
$2821109907456$ different permutations for 8 character passwords. 

\subsubsection*{b.}
Again, there are $36$ possible options for each character. The difference is,
for one of the characters, there are only $10$ possible options. Therefore,
there are $36^7\times10^1$, or $949318771330$ possible options for an 8
character password that requires at least one digit.  

\subsubsection*{c.}
We have the number of possible permutations in part \textit{a} of this problem,
the number of valid permutations in part \textit{b} of this problem, so, our
probability is as follows:

\[P(valid) = \frac{949318771330}{2821109907456}\]
\[P(valid) \approx 0.34 \]

\subsection*{Problem 11}
The probability that they match can be expressed as the union of two separate
probabilities, $P(2\;blue)$ and $P(2\;white)$. We can find them as follows:

\subsubsection*{$P(2\;blue)$}
\[P(2\;blue) = \frac{8}{14} \times \frac{4}{6}\]
\[P(2\;blue) = \frac{8}{21}\]

\subsubsection*{$P(2\;white)$}
\[P(2\;white) = \frac{6}{14} \times \frac{2}{6}\]
\[P(2\;white) = \frac{1}{7}\]

\subsubsection*{$P(same\;color)$}
\[P(same\;color) = \frac{8}{21} + \frac{1}{7}\]
\[P(same\;color) = \frac{11}{21}\]
\[P(same\;color) \approx 0.52 \]

\section*{Section 2.3}
\subsection*{Problem 1}
Based on what we discussed in class, $P(A)\times P(B)$ needs to equal 
$P(A \cap B)$. For this to be possible, $P(B)$ needs to be 
$\frac{0.2}{0.8}$ or $0.25$. 

\subsection*{Problem 2}
Based on what we discussed in class, $P(A) \times P(B^c)$ needs to equal 
$P(A \cap B^c)$. For this to be possible, $P(B^c)$ needs to be 
$\frac{0.4}{0.5}$ or $0.8$. Then we have that $P(B) = 1 - P(B^c)$, which
means $P(B) = 1 - 0.8 = 0.2$. 

\subsection*{Problem 5}
$P(B|A)$ is greater, because in many colleges, all engineers need to take
Calculus, but not all people taking Calculus need to take engineering. 

\subsection*{Problem 6}
The probability of the randomly selected person having an asthmatic attack on
the given day is $0.056 \times 0.027$ or $0.001512$. 

\subsection*{Problem 7}
\subsubsection*{a.}
The probability of both companies being profitable is $0.2 \times 0.15$ or
$0.03$.
\subsubsection*{b.}
The probability of neither company being profitable is 
$(1 - 0.2) \times (1 - 0.15)$ or $0.68$. 

\subsubsection*{c.}
The probability of at least one of the two companies being profitable is 
$1 - 0.68$ or $0.32$. 

\subsection*{Problem 8}
\subsubsection*{a.}
The probability that one of the two deploys is $1 - P(neither\;deploys)$. We can
find $P(neither\;deploys)$ from the main and backup probabilities like so:
\[P(neither\;deploys) = (1 - P(main)) \times (1 - P(backup)) \]
\[P(neither\;deploys) = (1 - 0.99) \times (1 - 0.98) \]
\[P(neither\;deploys) = 0.0002 \]

Thus, the probability that one of the two deploys is $1 - 0.0002$ or $0.9998$

\subsubsection*{b.}
The probability that the backup parachute deploys is the probability that the
main parachute doesn't deploy multiplied by the probability that the backup
chute does deploy:

\[P(backup\;deploys) = (1 - P(main)) \times P(backup) \]
\[P(backup\;deploys) = (1 - 0.99) \times 0.98 \]
\[P(backup\;deploys) = (1 - 0.99) \times 0.98 \]
\[P(backup\;deploys) = 0.0098 \]

\subsection*{Problem 17}
\subsubsection*{a.}
The probability is the number of valid outcomes over the number of total
outcomes, or $\frac{56+24}{100}$, which simplifies to $\frac{4}{5}$ or $0.8$. 

\subsubsection*{b.}
Similarly to part \textit{a}, the probability is $\frac{56+14}{100}$, which
simplifies to $\frac{7}{10}$ or $0.7$. 

\subsubsection*{c.}
The probability is as follows:

\[ P(1|2) = \frac{56}{56+24} \]
\[ P(1|2) = \frac{7}{10} \]
\[ P(1|2) = 0.7 \]

\subsubsection*{d.}
The genes are not in linkage equilibrium. Given the event that gene 2 is
dominant, there is a much higher likelihood that gene 1 being dominant, and vice
versa. 

\subsection*{Problem 21}
\subsubsection*{a.}
The assumption in the calculation is such that no external event will cause the
failure of both gauges at the same time, and that the failure of each gauge is
independent of the other. 

\subsubsection*{b.}
This assumption is likely not justified because we are specifically told that one 
potential cause of gauge failure is electrical cables being burned up, which
would affect both gauges, making their failures in some degree dependent on each
other. 

\subsubsection*{c.}
It is likely too low. It treats the gauges' failure as independent, where they
are actually dependent --- that means that $P(1|2) > 0$ and $P(2|1) > 0$. 

\subsection*{Problem 24}
\subsubsection*{a.}
\[ P(A) = \frac{300}{1000} \]
\[ P(A) = \frac{3}{10} \]
\[ P(A) = 0.3 \]

\subsubsection*{b.}
\[ P(B|A) = \frac{299}{999} \]
\[ P(B|A) \approx 0.30 \]

\subsubsection*{c.}
\[ P(A \cap B) = P(B|A) \times P(A) \]
\[ P(A \cap B) = \frac{299}{999} \times \frac{300}{1000} \]
\[ P(A \cap B) = \frac{299}{3330} \]
\[ P(A \cap B) \approx 0.090 \]

\subsubsection*{d.}
\[ P(A^c \cap B) = P(B|A^c) \times P(A^c) \]
\[ P(A^c) = 1 - P(A) \]
\[ P(A^c) = 1 - 0.3 \]
\[ P(A^c) = 0.7 \]
\[ P(B|A^c) = \frac{300}{999} \]
\[ P(B|A^c) \approx 0.30 \]
\[ P(A^c \cap B) \approx 0.3 \times 0.7 \]
\[ P(A^c \cap B) \approx 0.21 \]

\subsubsection*{e.}
$P(B)$ has two possibilities, the event $A^c$ happening, and the event $A$
happening. Then we average the two.

\[ P(B|A) = \frac{299}{999} \]
\[ P(B|A^c) = \frac{300}{999} \]
\[ P(B) = \frac{\frac{299}{999} + \frac{300}{999}}{2} \]
\[ P(B) \approx 0.30 \]


\subsubsection*{f.}
\[ P(B|A) = \frac{P(A \cap B)}{P(B)} \]
\[ P(B|A) = \frac{\frac{299}{3330}}{\frac{299}{999}} \]
\[ P(B|A) = \frac{3}{10} \]
\[ P(B|A) = 0.3 \]

\subsubsection*{g.}
A and B aren't independent, but for all intents and purposes they are. There is
a very minimal difference between $P(A)$ and $P(B)$ in all circumstances, so it
is reasonable to treat them as if they were independent, unless extremely
precise and exact answers were needed. 

\end{document}


