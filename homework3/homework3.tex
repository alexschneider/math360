%        File: homework3.tex
%     Created: Fri Jan 30 04:00 PM 2015 P
% Last Change: Fri Jan 30 04:00 PM 2015 P
%
\documentclass[11pt]{article}
\usepackage{geometry}
\geometry{letterpaper}
\usepackage[parfill]{parskip}
\title{Math 360 Homework 3}
\author{Alex Schneider}
\begin{document}
\maketitle
\section*{Section 2.2}
\subsection*{Problem 1}
\subsubsection*{a.}
Because this is ordered, with replacement, there are $4^3$ or $64$ different codons. 

\subsubsection*{b.}
We need to find how many codons there are who's first and third base is either
\textit{A} or \textit{G} which can be expressed as $2^2$ or $4$, and the second base is 
\textit{C} or \textit{T}, which can be expressed as $2^1$ or $2$. If we multiply
the two probabilities together, we get a total of $8$ different codons that meet
the requirements.

\subsubsection*{c.}
Because this is ordered, without replacement, there are $\frac{4!}{(4-3)!}$
different codons, which simplifies to $24$ different codons. 

\subsection*{Problem 4}
The way that we can solve this problem is by breaking it up into two parts.
First, the number of ways to divide 10 players into two teams. This is
unordered, without replacement, or $10 \choose 5$, which reduces to $252$
combinations. Then, each combination has two options, a red uniform or a blue
uniform. We multiply $252$ by $2$ to get $454$ different ways of dividing 10
basketball players in two teams. 

\subsection*{Problem 6}
In this circumstance, we have ordered, without replacement because the
president, vice president, and secretary cannot be the same person. We 
can express this by the formula $\frac{8!}{(8-3)!}$, which simplifies to $336$
different ways to chose a president, vice president, and secretary from an 8
person committee. 

\subsection*{Problem 9}
\subsubsection*{a.}
There are $26$ letters and $10$ digits, for a total of $36$ possible
options for each character. Since characters can be chosen more than once,
this is ordered, with replacement. Therefore, there are $36^8$, or
$2821109907456$ different permutations for 8 character passwords. 

\subsubsection*{b.}
Again, there are $36$ possible options for each character. The difference is,
for one of the characters, there are only $10$ possible options. Therefore,
there are $36^7\times10^1$, or $949318771330$ possible options for an 8
character password that requires at least one digit.  

\subsubsection*{c.}
We have the number of possible permutations in part \textit{a} of this problem,
the number of valid permutations in part \textit{b} of this problem, so, our
probability is as follows:

\[P(valid) = \frac{949318771330}{2821109907456}\]
\[P(valid) \approx 0.34 \]

\subsection*{Problem 11}
The probability that they match can be expressed as the union of two separate
probabilities, $P(2\;blue)$ and $P(2\;white)$. We can find them as follows:

\subsubsection*{$P(2\;blue)$}
\[P(2\;blue) = \frac{8}{14} \times \frac{4}{6}\]
\[P(2\;blue) = \frac{8}{21}\]

\subsubsection*{$P(2\;white)$}
\[P(2\;white) = \frac{6}{14} \times \frac{2}{6}\]
\[P(2\;white) = \frac{1}{7}\]

\subsubsection*{$P(same\;color)$}
\[P(same\;color) = \frac{8}{21} + \frac{1}{7}\]
\[P(same\;color) = \frac{11}{21}\]
\[P(same\;color) \approx 0.52 \]

\section*{Section 2.3}
\subsection*{Problem 1}
Based on what we discussed in class, $P(A)\times P(B)$ needs to equal 
$P(A \cap B)$. For this to be possible, $P(B)$ needs to be 
$\frac{0.2}{0.8}$ or $0.25$. 

\subsection*{Problem 2}
Based on what we discussed in class, $P(A) \times P(B^c)$ needs to equal 
$P(A \cap B^c)$. For this to be possible, $P(B^c)$ needs to be 
$\frac{0.4}{0.5}$ or $0.8$. Then we have that $P(B) = 1 - P(B^c)$, which
means $P(B) = 1 - 0.8 = 0.2$. 

\subsection*{Problem 5}
$P(B|A)$ is greater, because in many colleges, all engineers need to take
Calculus, but not all people taking Calculus need to take engineering. 

\subsection*{Problem 6}
The probability of the randomly selected person having an asthmatic attack on
the given day is $0.056 \times 0.027$ or $0.001512$. 

\subsection*{Problem 7}
\subsubsection*{a.}
The probability of both companies being profitable is $0.2 \times 0.15$ or
$0.03$.
\subsubsection*{b.}
The probability of neither company being profitable is 
$(1 - 0.2) \times (1 - 0.15)$ or $0.68$. 

\subsubsection*{c.}
The probability of at least one of the two companies being profitable is 
$1 - 0.68$ or $0.32$. 

\subsection*{Problem 8}
\subsubsection*{a.}
The probability that one of the two deploys is 

\subsubsection*{b.}


\subsection*{Problem 17}

\subsection*{Problem 21}

\subsection*{Problem 24}

\end{document}


