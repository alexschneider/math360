%        File: homework5.tex
%     Created: Fri Feb 06 04:00 PM 2015 P
% Last Change: Fri Feb 06 04:00 PM 2015 P
%\documentclass[11pt]{article}
\documentclass[11pt]{article}
\usepackage{geometry}
\usepackage{cancel}
\geometry{letterpaper}
\usepackage[parfill]{parskip}
\title{Math 360 Homework 5}
\author{Alex Schneider}
\begin{document}
\maketitle

\section*{2.4}
\subsection*{Problem 15}
\subsubsection*{a.}
The mean lifetime can be found as follows:

\[ \mu_x = \int_0^\infty t \times 0.1e^{-0.1t} dt \]
\[ \mu_x = uv - \int_{0}^{\infty} v du \]
\[
    \begin{array}{ll}
        u = t & dv = 0.1e^{-0.1t} dt \\
        du = dt & v = -e^{-0.1t}
    \end{array} 
\]
\[ \mu_x = t\times -e^{-0.1t} \bigg\vert^\infty_0 - \int_{0}^{\infty} -e^{-0.1t} dt \]
\[ \mu_x = t\times -e^{-0.1t}  - 10e^{-0.1t} \bigg\vert^\infty_0 \]
\[ \mu_x = \left[ -e^{-0.1t} \left( t + 10  \right) \right] \bigg\vert^\infty_0\]
\[ \mu_x = \left[ \cancelto{0}{-e^{-0.1\times \infty}} \left( \infty + 10  \right) \right] - 
           \left[ -e^{-0.1\times 0} \left( 0 + 10  \right) \right] \]
\[ \mu_x = - \left[ -1 \left( 10 \right) \right] \]
\[ \mu_x = 10 \]

\subsubsection*{b.}
The standard deviation of the lifetimes is as follows:
\[ \sigma^2_x = \int_0^{\infty}{t^2 \times 0.1e^{-0.1t}}dt - 10^2 \]
\[ \sigma^2_x = uv - \int_{0}^{\infty} v du - 100 \]
\[
    \begin{array}{ll}
        u = t^2 & dv = 0.1e^{-0.1t} dt \\
        du = 2t\, dt & v = -e^{-0.1t}
    \end{array} 
\]
\[ \sigma^2_x = t^2\times \left( -e^{-0.1t} \right) \bigg\vert^\infty_0 -
                \int_{0}^{\infty} -e^{-0.1t} \times 2t \, dt - 100 \]
\[
    \begin{array}{ll}
        u = 2t & dv = -e^{-0.1t} dt \\
        du = 2\, dt & v = 10e^{-0.1t}
    \end{array} 
\]
\[ \sigma^2_x = t^2\times \left( -e^{-0.1t} \right) \bigg\vert^\infty_0 -
                \left( 
                    2t \times 10e^{-0.1t} \bigg\vert^\infty_0 - 
                    2 \times \int_{0}^{\infty} 10e^{-0.1t} dt
                \right) - 100\]
\[ \sigma^2_x = 
        {\left[ 
            t^2\times \left( -e^{-0.1t} \right)  -
            2 \times \left( 
                t \times 10e^{-0.1t} - 
                \left( -
                    100e^{-0.1t}
                \right)
            \right)
        \right]}^\infty_0 - 100\]
\[ \sigma^2_x = 
        {\left[ -e^{-0.1t} \left( 
            t^2  + 20t + 200
        \right)\right]}^\infty_0 - 100 \]
\[ \sigma^2_x = 
    \cancelto{0}{-e^{-0.1\times \infty}} \left( 
            \infty^2  + 20\times \infty + 200
        \right) - 
        \left(-\cancelto{1}{e^{-0.1\times 0}} \left( 
        \cancelto{0}{0^2  + 20\times 0} + 200
        \right) \right)
    - 100 \]
\[ \sigma^2_x = 200 - 100 \]
\[ \sigma^2_x = 100 \]
\[ \sigma_x = 10 \]

\subsubsection*{c.}
The cumulative distribution function of the lifetime is as follows:

\[ F(x) = \int_{0}^{t}0.1e^{-0.1x}dx \]
\[ F(x) = {\left[ -e^{-0.1x} \right]}^t_0 \]
\[ F(x) = -e^{-0.1t} - \left( -\cancelto{1}{e^{-0.1\times 0}} \right) \]
\[ F(x) = -e^{-0.1t} + 1 \]

\subsubsection*{d.}
The probability is $F(12)$, or about $0.699$. 

\subsection*{Problem 16}
\subsubsection*{a.}
The mean diameter is as follows:

\[ \mu_x = \int_{9.75}^{10.25}x \times 3\left[ 1 - 16{\left( x - 10 \right)}^2 \right]dx \]
\[ \mu_x = 3 \times \int_{9.75}^{10.25}x \left[ 1 - 16 \left(x^2 - 20x + 100 \right) \right]dx \]
\[ \mu_x = 3 \times \int_{9.75}^{10.25}x \left[ - 16x^2 + 320x - 1599 \right]dx \]
\[ \mu_x = 3 \times \int_{9.75}^{10.25}-16x^3 + 320x^2 - 1599x dx \]
\[ \mu_x = 3 {\left[ -4x^4 + \frac{320}{3}x^3 - \frac{1599}{2} x^2 \right]}^{10.25}_{9.75} \] 
\[ \mu_x = 3 \left[ -4\times 10.25^4 + \frac{320}{3}\times 10.25^3 - \frac{1599}{2} \times 10.25^2 \right] -
             \left[ -4\times 9.75^4 + \frac{320}{3}\times 9.75^3 - \frac{1599}{2} \times 9.75^2 \right]\] 
\[ \mu_x = 10 \]

\subsubsection*{b.}
\[ \sigma_x^2 = \int_{9.75}^{10.25} {(x - 10)}^2 \times 3\left[ 1 - 16{(x - 10)}^2 \right]dx \]
\[ \sigma_x^2 = 3 \times \int_{9.75}^{10.25} {(x - 10)}^2 \times \left[ 1 - 16{(x - 10)}^2 \right]dx \]
\[ \sigma_x^2 = 3 \times \int_{9.75}^{10.25} {(x - 10)}^2 dx  - 16\times\int_{9.75}^{10.25} {(x - 10)}^4  dx \]
\[ \sigma_x^2 = {\left[ {(x-10)}^3 \right]}^{10.25}_{9.75}  - 16 \times {\left[\frac{{(x - 10)}^5}{5} \right]}^{10.25}_{9.75}\]
\[ \sigma_x^2 = 0.0125 \]
\[ \sigma_x \approx 0.1118 \]

\subsubsection*{c.}
\[ F(X) = \int_{9.25}^{a} 3\left[ 1-16{\left( a-10 \right)}^2 \right] da \]
\[ F(X) = 3 - 3 \times 16 \times \int_{9.25}^{x} {\left( a-10 \right)}^2 da \]
\[ F(X) = 3 - 16 \times {\left[ {(a-10)}^3 \right]}^{x}_{9.75} \]
\[ F(X) = 3 - 16 \times \left[ {(x-10)}^3 - {(9.75-10)}^3 \right] \]
\[ F(X) = 3 - 16 \times \left[ {(x-10)}^3 - {(-0.25)}^3 \right] \]
\[ F(X) = 3 - 16 \times \left[ {(x-10)}^3 + 0.015625 \right] \]
\[ F(X) = 3 - 16{(x-10)}^3 - 0.25 \]
\[ F(X) = 2.75 - 16{(x-10)}^3 \]



\subsubsection*{d.}
The proportion of piston rings that have diameters less than $9.75 cm$ is $0$
because the minimum tolerance is $9.75 cm$. 

\subsubsection*{e.}
The proportion of piston rings that have diameters $9.75 cm$ and $10.25 cm$ is
$1$ because the maximum bounds of the tolerance are those two numbers. 

\subsection*{Problem 18}
\subsubsection*{a.}
\[ P(X \ge 3) = 81 \times \int_{3}^{\infty} \frac{1}{{(x+3)}^4} dx \]
\[ u = x + 3 \]
\[ du = dx \]
\[ P(X \ge 3) = 81 \times \int_{6}^{\infty} \frac{1}{{(u)}^4} du \]
\[ P(X \ge 3) = \lim_{b \to \infty} \left( -\frac{27}{u^3} \right) \bigg \vert^b_6\]
\[ P(X \ge 3) = \cancelto{0}{\lim_{b \to \infty} \left( -\frac{27}{b^3} \right)}
              - \left( -\frac{27}{6^3} \right)\]

\[ P(X \ge 3) = \frac{1}{8} \]
\subsubsection*{b.}
\[ P(1 \le X \le 3) = 81 \times \int_{1}^{3} \frac{1}{{(x+3)}^4} dx \]
\[ u = x + 3 \]
\[ du = dx \]
\[ P(1 \le X \le 3) = 81 \times \int_{6}^{4} \frac{1}{{(u)}^4} du \]
\[ P(1 \le X \le 3) = \left(- \frac{27}{u^3} \right) \bigg \vert_4^6 \]
\[ P(1 \le X \le 3) = \left(- \frac{27}{6^3} \right) - \left(- \frac{27}{4^3} \right) \]
\[ P(1 \le X \le 3) = \frac{19}{24} \]
\[ P(1 \le X \le 3) \approx 0.2968 \]

\subsubsection*{c.}
\[ \mu_x = 81 \times \int_{0}^{\infty} \frac{x}{{(x+3)}^4} dx \]
\[
    \begin{array}{ll}
        u = x & dv = \frac{1}{{(x+3)}^4} \\
        du = dx & v = \frac{1}{-3{(x+3)}^3}
    \end{array} 
\]
\[ \mu_x = 81 \times \frac{x}{-3{(x+3)}^3} \times
            \int_{0}^{\infty} \frac{1}{-3{(x+3)}^3} dx \]
\[ \mu_x = 81 \times { \left[ \frac{x}{-3{(x+3)}^3} \times
    \frac{1}{-6{(x+3)}^2}\right] }_0^{\infty} \]
\[ \mu_x = 81 \times \cancelto{0}{\lim_{b \to \infty}  
        \left[ \frac{b}{-3{(b+3)}^3} \times \frac{1}{-6{(b+3)}^2}\right]
    } - {
        \left[ \frac{0}{-3{(0+3)}^3} \times \frac{1}{-6{(0+3)}^2}\right] 
    }
\]
\[ \mu_x = \frac{3}{2} \]

\subsubsection*{d.}
\[ \sigma_x^2 = \int_{0}^{\infty} \frac{x^2}{{(x+3)}^4} dx - \frac{3}{2} \]
\[ \sigma_x^2 = \frac{15}{2} \]


\subsubsection*{e.}
\[ F(X) = 81 \times \int_{0}^{x} \frac{1}{{(a+3)}^4} da \]
\[ F(X) = 1 - \frac{1}{3(x+3)^3} \]

\subsubsection*{f.}
\[ 0.5 = 81 \times \int_{0}^{x_m} \frac{1}{{(a+3)}^4} da \]
\[ x_m \approx 0.7798 \]

\subsubsection*{g.}
\[ 0.3 = 81 \times \int_{0}^{x_m} \frac{1}{{(a+3)}^4} da \]
\[ x_m \approx 0.3787 \]

\subsection*{Problem 21}
\subsubsection*{a.}
\[ P(X\ge0.5) = \int_{0.5}^{1} 1.2(x+x^2) dx \]
\[ P(X\ge0.5) = 0.8 \]
\subsubsection*{b.}
\[ \mu_x = \int_{0}^{1} 1.2x(x+x^2) dx \]
\[ \mu_x = 0.7 \]

\subsubsection*{c.}
\[ P(0.6\le X\le0.8) = \int_{0.6}^{0.8} 1.2(x+x^2) dx \]
\[ P(0.6\le X\le0.8) = 0.2864 \]

\subsubsection*{d.}
\[ \sigma_x^2 = \int_{0}^{1} 1.2x^2(x+x^2) dx - 0.7^2 \]
\[ \sigma_x^2 = 0.05 \]
\[ \sigma_x = 0.2236 \]

\subsubsection*{e.}
\[ P(0.7 - 0.4472 \le X\le 1) = \int_{0.7 - 0.4472}^{1} 1.2(x+x^2) dx \]
\[ P(0.7 - 0.4472 \le X\le 1) = 0.9552 \]


\subsubsection*{f.}
\[ F(X) = \int_{0}^{x} 1.2(a+a^2) da \]
\[ F(X) = (0.4x + 0.6) x^2  \]

\subsection*{Problem 25}
\subsubsection*{a.}
\[P(X\le2) = \int_{0}^{2}xe^{-x} dx \]
\[P(X\le2) = 0.5940 \]

\subsubsection*{b.}
\[ P(1.5 \le X \le 3) = \int_{1.5}^{3}xe^{-x} dx \]
\[ P(1.5 \le X \le 3) = 0.3587 \]

\subsubsection*{c.}
\[ \mu_x = \int_{0}^{\infty}x^2e^{-x} dx \]
\[ \mu_x = 2 \]

\subsubsection*{d.}
\[P(X\le2) = \int_{0}^{x}ae^{-a} da \]
\[P(X\le2) = 1 - e^{-x}(x+1) \]

\section*{2.5}
\subsection*{Problem 1}
\subsubsection*{a.}
The mean of $3X$ is $3\mu_x$ or $3\times 9.5$, which simplifies to $28.5$
because we are applying a linear shift to $X$.

The standard deviation of $3X$ is $|3|\sigma_x$ or $3\times 0.4$, or $1.2$,
because we are applying a linear shift to $X$. 

\subsubsection*{b.}
The mean of $Y - X$ is $\mu_Y - \mu_X$, or $6.8 - 9.5$, which simplifies to
$-2.7$. 

The standard deviation of $Y - X$ is $\sigma_Y + \sigma_X$, or $0.1+0.4$, which
simplifies to $0.5$. This is because the standard deviation is the square root
of a squared number, which cancels out the negative. 

\subsubsection*{c.}
The mean of $4Y$ is $\mu_Y\times 4$, or $6.8 \times 4$, which simplifies to
$27.2$. The mean of $X + 4Y$ is $\mu_X + 27.2$, or $9.5 + 27.2$, which
simplifies to $36.7$. 

The standard deviation of $4Y$ is $|4|\times \sigma_Y$, or $0.4$. The standard
deviation of $X + 4Y$ is $\sigma_X + 0.4$, or $0.4+0.4$, which simplifies to
$0.8$. 

\subsection*{Problem 7}
\subsubsection*{a.}
$\mu_M$ is $\mu_Y+1.5\times \mu_Y$, or $0.350 + 1.5\times 0.125$, which
simplifies to $0.5375$.

\subsubsection*{b.}
$\sigma_M$ is $\sigma_Y+|1.5|\times \sigma_Y$, or $0.05+|1.5|\times 0.10$. This
simplifies to $0.2$

\subsection*{Problem 8}
\subsubsection*{a.}
The mean of the total weight is $12.02 \times 12 oz$ or $144.24 oz$.

\subsubsection*{b.}
The standard deviation of the total weight is $|12|\times 0.03 oz$, or $0.36 oz$.

\subsubsection*{c.}
The mean of the average weight per box of the cereal is $12.02 oz$ as given in the
problem description.

\subsubsection*{d.}
The standard deviation per box of cereal is $0.03 oz$ as given in the problem
description. 

The 


\subsection*{Problem 12}
\end{document}


