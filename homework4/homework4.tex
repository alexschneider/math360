%        File: homework4.tex
%     Created: Tue Feb 03 05:00 PM 2015 P
% Last Change: Tue Feb 03 05:00 PM 2015 P
%
\documentclass[11pt]{article}
\usepackage{geometry}
\geometry{letterpaper}
\usepackage[parfill]{parskip}
\title{Math 360  Homework 4}
\author{Alex Schneider}
\begin{document}
\maketitle
\section*{Section 2.3}
\subsection*{29}
\subsubsection*{a.}
The probability that an item with a flaw will not be detected by the first
inspector is $0.1$. This means that if a flaw shows up with a probability of
$0.10$, the probability of an item passing by the first inspector with a flaw is
$0.1 \times 0.1$, or $0.01$. 

\subsubsection*{b.}
We have that probability of the item passing by the first inspector with a flaw is $0.01$. The
second inspector has a probability of not detecting the flaw of $0.3$.
Therefore, the probability an item having an undetected flaw is $0.01 \times 0.3$,
or $0.003$. 

\subsection*{30}
\subsubsection*{a.}
The probability of someone testing positive and having a disease is as follows:

\[P(D|+) = \frac{(0.99)(0.05)}{(0.99)(0.05)+(0.01)(0.95)}\]
\[P(D|+) = \frac{(0.99)(0.05)}{(0.99)(0.05)+(0.01)(0.95)}\]
\[P(D|+) \approx 0.84\]

\subsubsection*{b.}
The probability of someone testing negative and not having the disease is as
follows:
\[P(D^c|-) = \frac{P(-|D^c)P(D^c)}{P(-|D^c)P(D^c)+P(-|D)P(D)} \]
\[P(D^c|-) = \frac{(0.99)(0.95)}{(0.99)(0.95)+(0.01)(0.05)} \]
\[P(D^c|-) \approx 0.999 \]

\subsection*{31}
\subsubsection*{a.}
The probability of one child not having the disease is $0.75$. Therefore, the
probability of two children not having the disease is $0.75^2$, or $0.5625$. 

\subsubsection*{b.}
The probability that one child is a carrier is $0.5$. Therefore, the probability
of both children being carriers is $0.5^2$, or $0.25$. 

\subsubsection*{c.}
The probability of both children being carriers given that neither have the
disease is as follows:

\[P(C) = 0.25 \]
\[P(D^c) = 0.5625 \]
\[P(C \cap D^c) = 0.25 \]
\[P(C|D^c) = \frac{P(C \cap D^c)}{P(D^c)} \]
\[P(C|D^c) = \frac{0.25}{0.5625} \]
\[P(C|D^c) \approx 0.\overline{4}  \]

\subsubsection*{d.}
The probability of the woman being a carrier is $0.5$. The probability that the
child will have the disease if the woman is a carrier is $0.25$. Therefore, the
probability that the child will have the disease is $0.5 \times 0.25$ or
$0.125$. 

\subsection*{33}
33. Refer to Example 2.26.
a. If a man tests negative, what is the probability that
he actually has the disease?
b. For many medical tests, it is standard procedure to
repeat the test when a positive signal is given. If
repeated tests are independent, what is the proba-
bility that a man will test positive on two successive
tests if he has the disease?
c. Assuming repeated tests are independent, what is
the probability that a man tests positive on two
successive tests if he does not have the disease?
d. If a man tests positive on two successive tests, what
is the probability that he has the disease?
\subsubsection*{a.}
If a man tests negative, he has a probability as follows:

\[ P(D|-) = \frac{P(-|D)P(D)}{P(-|D)P(D)+P(-|D^c)P(D^c)} \]
\[ P(D|-) = \frac{(0.01)(0.005)}{(0.01)(0.005) + (0.01)(0.995)} \]
\[ P(D|-) = 0.005 \]

\subsubsection*{b.}
The probability of testing positive on two tests given that he has the disease
is $0.99^2$ or $0.9801$.

\subsubsection*{c.}
The probability of testing positive on two successive tests if he doesn't have
the disease is $0.01^2$ or $0.0001$. 

\subsubsection*{d.}
The probability is as follows:

\[ P(D|+^2) = \frac{P(+^2|D)P(D)}{P(+^2|D)P(D)+P(+^2|D^c)P(D^c)} \]
\[ P(D|+^2) = \frac{(0.9801)(0.005)}{(0.9801)(0.005)+(0.0001)(0.995)} \]
\[ P(D|+^2) \approx 0.9801  \]

\section*{Section 2.4}
\subsection*{3}
\subsubsection*{a.}
The mean number is as follows:

\[ \mu_x = 1(0.4)+2(0.2)+3(0.2)+4(0.1)+5(0.1) \]
\[ \mu_x = 2.3 \]

\subsubsection*{b.}
The variance is as follows:

\[ \sigma^2_x = \sum_x {(x-2.3)}^2 P(X=x) \]
\[ \sigma^2_x = {(1-2.3)}^2 (0.4) + {(2-2.3)}^2(0.2) + {(3-2.3)}^2(0.2) +
                {(4-2.3)}^2(0.1) + {(5-2.3)}^2(0.1) \]
\[ \sigma^2_x = 1.81 \]

\subsubsection*{c.}
The standard deviation is as follows:

\[ \sigma_x = \sqrt{\sigma^2_x} \]
\[ \sigma_x \approx \sqrt{1.81} \]
\[ \sigma_x \approx 1.35 \]

\subsubsection*{d.}
The probability mass function is as follows:

\begin{tabular}{r|ccccc} % chktex 44
    x & $1 \times Y$ & $2 \times Y$ & $3 \times Y$ & $4 \times Y$ & $5 \times Y$ \\
    \hline % chktex 44
    $p(x)$ & 0.4 & 0.2 & 0.2 & 0.1 & 0.1
\end{tabular}

\subsubsection*{e.}
The mean number of gallons ordered would be $2.3 \times Y$. 

\subsubsection*{f.}
The variance of the number of gallons ordered would be $1.81 \times Y$. 

\subsubsection*{g.}
The standard deviation of the number of gallons ordered is approximately 
$1.35 \times Y$. 

\subsection*{4}
\subsubsection*{a.}
$p_3(x)$ is the only function that all the probabilities add up to 1. Therefore,
it must be the probability mass function of X.

\subsubsection*{b.}
$\mu_x$ is as follows:

\[ \mu_x = 0(0.2)+1(0.2)+2(0.3)+3(0.1)+4(0.1) \]
\[ \mu_x = 1.5 \]

$\sigma_x^2$ is as follows:

\[ \sigma_x^2 = (0.2){(0-1.5)}^2 + (0.2){(1-1.5)}^2 + (0.3){(2-1.5)}^2 + (0.1){(4-1.5)}^2 + (0.1){(5-1.5)}^2 \]
\[ \sigma_x^2 = 2.425 \]

\subsection*{5}
\subsubsection*{a.}
The probability mass function is as follows:

\begin{tabular}{r|ccccc} % chktex 44
    x & 1 & 2 & 3 & 4 & 5 \\
    \hline % chktex 44
    $p(x)$ & 0.70 & 0.15 & 0.10 & 0.03 & 0.02
\end{tabular}

\subsubsection*{b.}
\[P(X \le 2) = 0.70 + 0.15 \]
\[P(X \le 2) = 0.85 \]

\subsubsection*{c.}
\[P(X > 3) = 0.03 + 0.02 \]
\[P(X > 3) = 0.05 \]

\subsubsection*{d.}
$\mu_x$ is as follows:

\[ \mu_x = 1(0.70)+2(0.15)+3(0.10)+4(0.03)+5(0.02) \]
\[ \mu_x = 1.52 \]

\subsubsection*{e.}
$\sigma_x$ is as follows:

\[ \sigma_x^2 = (0.70){(1 - 1.52)}^2 + (0.15){(2 - 1.52)}^2 + (0.10){(3 - 1.52)}^2 +
                (0.03){(4 - 1.52)}^2 + (0.02){(5 - 1.52)}^2 \]

\[ \sigma_x^2 = 0.8696 \]
\[ \sigma_x = \sqrt{0.8696} \]
\[ \sigma_x \approx 0.93 \]

\subsection*{7}
\subsubsection*{a.}
For $p(x)$ to be a probability mass function, $p(1)+p(2)+p(3)+p(4)$ needs to
equal $1$. 

\[ 1 = p(1)+p(2)+p(3)+p(4) \]
\[ 1 = 1c+2c+3c+4c \]
\[ 1 = 10c \]
\[ \frac{1}{10} = c \]
\[ 0.1 = c \]

For $p(x)$ to be a probability mass function, $c$ needs to be $0.1$. 

\subsubsection*{b.}
\[ P(X=2) = 0.1 \times 2 \]
\[ P(X=2) = 0.2 \]

\subsubsection*{c.}
The mean number of properly function components is as follows:

\[ \mu_x = 1(0.1)+2(0.2)+3(0.3)+4(0.4) \]
\[ \mu_x = 3 \]

\subsubsection*{d.}
The variance is as follows:

\[ \sigma_x^2 = (0.1){(1-3)}^2 + (0.2){(2-3)}^2 + (0.3){(3-3)}^2 + (0.4){(4-3)}^2 \]
\[ \sigma_x^2 = 1.0 \]

\subsubsection*{e.}
The standard deviation is as follows:

\[ \sigma_x = \sqrt{\sigma_x^2} \]
\[ \sigma_x = \sqrt{1.0} \]
\[ \sigma_x = 1.0 \]

\subsection*{10}
\subsubsection*{a.}
The probability of the first chip chosen being acceptable is $0.9$. 

\subsubsection*{b.}
The probability that the first chip is unacceptable and the second is acceptable
is $0.1 \times 0.9$ or $0.09$. 

\subsubsection*{c.}
The probability of the third chip being the first acceptable chip is the
probability of the first chip being unacceptable and the probability in
\textit{b.} This is $0.1 \times 0.09$ or $0.009$. 

\subsubsection*{d.}
The probability mass function is as follows:
\[ p(x) = \left\{
    \begin{array}{ll}
        9 \times 10^{-x} & x = 1, 2, 3, \ldots \\
        0 & otherwise
    \end{array}
    \right.
\]

\subsection*{11}
\subsubsection*{a.}
The smallest possible value for \textit{Y} would be 2, because that assumes the
first two chips are acceptable. 

\subsubsection*{b.}
The probability that \textit{Y} is 2 is $0.9 \times 0.9$ or $0.81$.

\subsubsection*{c.}
$P(Y=3|X=1)$ is as follows:

\[ P(Y=3|X=1) = \frac{P(Y=3 \cap X=1)}{P(X=1)} \]
\[ P(Y=3 \cap X=1) = P(X=1) \times P(X^c=2) \times P(X=3) \]
\[ P(Y=3 \cap X=1) = 0.9 \times 0.1 \times 0.9 \]
\[ P(Y=3 \cap X=1) = 0.081 \]
\[ P(Y=3|X=1) = \frac{0.081}{0.9} \]
\[ P(Y=3|X=1) = 0.09 \]

\subsubsection*{d.}
\[ P(Y=3|X=2) = \frac{P(Y=3 \cap X=2)}{P(X=2)} \]
\[ P(Y=3 \cap X=2) = P(X^c=1) \times P(X=2) \times P(X=3) \]
\[ P(Y=3 \cap X=2) = 0.1 \times 0.9 \times 0.9 \]
\[ P(Y=3 \cap X=2) = 0.081 \]
\[ P(Y=3|X=2) = \frac{0.081}{0.9} \]
\[ P(Y=3|X=2) = 0.09 \]

\subsubsection*{e.}
\[ P(Y=3) = P(Y=3|X=1) + P(Y=3|X=2) \]
\[ P(Y=3) = 0.09 + 0.09 \]
\[ P(Y=3) = 0.18 \]

\end{document}


