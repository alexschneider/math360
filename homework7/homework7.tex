%        File: homework7.tex
%     Created: Fri Mar 06 08:00 AM 2015 P
% Last Change: Fri Mar 06 08:00 AM 2015 P
%
\documentclass[11pt]{article}
\usepackage{geometry}
\usepackage{gensymb}
\geometry{letterpaper}
\usepackage[parfill]{parskip}
\title{Math 360 Homework 7}
\author{Alex Schneider}
\begin{document}
\maketitle
\section{Handout}
\subsection{1}
\[ \hat{B_1} = \frac{
    0\times1+1\times1+2\times2+3\times3+4\times3+5\times4 - 6\times2.5\times2.3
}{0^2+1^2+2^2+3^2+4^2+5^2+6\times2.5} = 0.19 \]

\[ \hat{B_0} = 2.5 - 0.19\times2.3 = 2.06 \]

\[ y = 2.06 + 0.19x \]

For an $x$ value of $10\%$, we can expect a $y$ value of about $7.05$
seconds.

\subsection{2}
\subsubsection{a.}
The USA is towards the lower right hand side of the scatter plot. 

\subsubsection{b.}
Sudan will likely have a number of TVs per household of about $44.5$.

\subsubsection{c.}
The life expectancy in Ethiopia is likely $-227.898$ years. 

\subsubsection{d.}
Correlation doesn't imply causation - as shown by Ethiopia's extrapolated life
expectancy. More likely, the correlation was caused by the two bits of data
being externally linked - that is, they both are related to the wealth and
prosperity of the respective country. 

\section{7.1}
\subsection{2}
\subsubsection{a.}
All the $y$ values are shifted up by 3. 

\subsubsection{b.}
All the $x$ values are multiplied by 10 and shifted by 1 in addition to the $y$
shift. 

\subsubsection{c.}
The $y$ values are the same, but in a different order, and the corresponding $x$
values are multiplied by 10 and shifted by 3. 

\subsection{3}
\subsubsection{a.}
The coefficient is a somewhat decent summary. There is a mostly strong positive
correlation, but it's not great at estimating the individual values. 
\subsubsection{b.}
There is clearly a relationship between $x$ and $y$, but linear regression is
not capable of representing it, so it appears to have a weak correlation. It is
not a good representation. 

\subsubsection{c.}
Even though there are a couple outliers, the coefficient does a great job at
representing the range of points. 

\subsection{4}
\subsubsection{a.}
False - for a positive correlation, the above average values must be
complemented by both variables. 

\subsubsection{b.}
False - for a negative correlation, the above average of one variable must be
matched by below average of another variable.

\subsubsection{c.}
False - The actual values don't matter at all - what matters is the values
relative to the other values. 

\subsection{5}
The correlation will likely be less than 0.6. The reason being that for women
being the same height as men, they're statistically lighter weight, so the data
sets won't line up exactly. 
 
\section{7.2}
\subsection{1}
\subsubsection{a.}
The number of pounds consumed when the temperature is $65 \degree C$ is 319.27
pounds. 
\subsubsection{b.}
I would expect the weight to differ by $1.13\times5$ or $5.65$ pounds. 

\subsection{2}
\subsubsection{a.}
The tensile strength is $52.214 ksi$. 

\subsubsection{b.}
I would expect the tensile strength to differ by $2.42\times3 ksi$ or $7.26
ksi$. 

\subsection{7}
\[ \hat{B_1} = 0.85 \frac{1.9}{1.2} = 1.34583 \]
\[ \hat{B_0} = 30.4  - 1.34583 \times 8.1 = 19.4988 \] 
The least squares line is:
\[ \hat{y} = 19.4988 - 1.34583 \hat{x}  \]

\subsection{12}
\[ \hat{B_1} = 0.7 \frac{100}{2} = 35 \]
\[ \hat{B_0} = 1350 - 35 \times 5  = 1175 \]
The least squares line is:
\[ \hat{y} = 1175 - 35 \hat{x} \]
 
\subsection{15}
\textit{iv.} We cannot tell unless we know the correlation. 

\end{document}


